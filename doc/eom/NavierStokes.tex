\section{The Navier-Stokes Momentum Equation}
\begin{rk}
The Navier-Stokes equations, developed in the ninteenth
century by French engineer and physicist \emph{Claude-Louis Navier}
and British physicist and mathematician
\emph{Sir George Gabriel Stokes}, describe the the viscous motion
of isotropic Newtonian fluids.  The notion of a Newtonian fluid is named
after English mathematician \emph{Sir Issac Newton} who made the
observation that the deviatoric stress in many fluids can be approximated
to be proportional to the gradient of velocity.
\end{rk}

\begin{df}[Static Pressure]
    The flow property \emph{static pressure} $p:\R^4\to \R$ maps
    \begin{equation}
        (t,r)\mapsto -\frac{1}{3} \tr \sigma(t,r).
    \end{equation}
\end{df}

\begin{df}[Deviatoric Stress Tensor]
The \emph{deviatoric stress tensor} $\tau:\R^4\to M_3(\R)$
maps
    \begin{equation}
        (t,r) \mapsto \sigma(t,r) + p(t,r) I_3.
    \end{equation}
\end{df}

\begin{rk}
The deviatoric stress tensor is the deviation of the Cauchy stess tensor
from the \emph{mean hydrostatic stress tensor} $-p I_3$ also called the 
\emph{volumetric stress tensor}.
\end{rk}

\begin{thm}[Trace of Deviatoric Stress Tensor is Zero]
    \begin{equation}
        \Tr \tau = 0.
    \end{equation}
\end{thm}
\begin{proof}
    \begin{align}
        \Tr \tau &= \Tr \sigma + \Tr p I_3 \\
        \Tr \tau &= \Tr \sigma + 3 p \\
        \Tr \tau &= \Tr \sigma + 3 \left( -\frac{1}{3} \Tr \sigma \right) \\
        \Tr \tau &= \Tr \sigma - \Tr \sigma = 0.
    \end{align}
\end{proof}

\begin{df}[Newtonian Fluid]
    A fluid is \emph{Newtonian} iff the deviatoric stress tensor is a linear function of the velocity gradient.
    That is, there exist coefficients
    $L_{ijkl}\in \R$ such that
    \begin{equation}
        \tau_{ij} = \sum_{k=0}^2 \sum_{l=0}^2 L_{ijkl} (v\tp{\ldel_r})_{kl}.
    \end{equation}
    The notation $(\cdot)_{ij}$ denotes the element at the $i^\text{th}$ row and $j^\text{th}$ column of matrix $(\cdot)$.
\end{df}

\begin{rk}
A fluid is isotropic if it has the same physical properties when measured
in all directions.
\end{rk}

\begin{df}[Isotropic Newtonian Fluid]
    A Newtonian fluid is \emph{isotropic} iff for every rotation matrix $R\in \SO(3;\R)$,
    \begin{equation}
        (R\tau \tp{R})_{ij} = \sum_{k=0}^2 \sum_{l=0}^2 L_{ijkl} (R(v\tp{\ldel_r})\tp{R})_{kl}.
    \end{equation}
\end{df}

\begin{thm}[Stokes Relations]
    Let the \emph{shear viscosity} $\mu:\R^4\to \R$ be a flow property.
    The deviatoric stress tensor of an isotropic Newtonian fluid must be of the form
    \begin{equation}
        \tau_{ij} = \begin{cases}
            2 \mu \pd{v_i}{r_i} - \frac{2}{3} \mu \tr( v \tp{\ldel_r} ) &\text{ for $i=j$ } \\
            \mu \left( \pd{v_i}{r_j} + \pd{v_j}{r_i} \right) &\text{ for $i\neq j$. }
        \end{cases}
    \end{equation}
\end{thm}
\begin{proof}
    \begin{equation}
        \tau_{ij} = \sum_{k=0}^2 \sum_{l=0}^2 L_{ijkl} (v\tp{\ldel_r})_{kl}.
    \end{equation}
    Let $L_{kl}\in M_3(\R)$ denote the matrix with entries $(L_{kl})_{ij} = L_{ijkl}$.  Then
    \begin{equation}
        \tau = \sum_{k=0}^2 \sum_{l=0}^2 L_{kl} (v\tp{\ldel_r})_{kl}.
    \end{equation}
    Because the stress tensor is symmetric, $L_{ijkl}=L_{jikl}$ and so $L_{kl}$ is symmetric.
    The fluid is isotropic, therefore for every $R\in \SO(3;\R)$,
    \begin{align}
        R \tau \tp{R} &= \sum_{k=0}^2 \sum_{l=0}^2 L_{kl} (R (v \tp{\ldel_r}) \tp{R})_{kl} \\
        \tau &= \sum_{k=0}^2 \sum_{l=0}^2 \tp{R} L_{kl} R (R (v \tp{\ldel_r}) \tp{R})_{kl} \\
        \underbrace{ \sum_{k=0}^2 \sum_{l=0}^2 L_{kl} (v \tp{\ldel_r})_{kl}}_{\tau^L} &=
            \underbrace{\sum_{k=0}^2 \sum_{l=0}^2 \tp{R} L_{kl} R (R (v \tp{\ldel_r}) \tp{R})_{kl}}_{\tau^R(R)}.
    \end{align}
    Choosing
    \begin{equation}
        R_0 = \begin{bmatrix}
            1 & 0 & 0 \\
            0 & -1 & 0 \\
            0 & 0 & -1
        \end{bmatrix},
    \end{equation}
    gives
    \begin{align}
        \tp{R_0} L_{kl} R_0 &= \begin{bmatrix}
            L_{00kl} & -L_{01kl} & -L_{02kl} \\
           -L_{10kl} &  L_{11kl} &  L_{12kl} \\
           -L_{20kl} &  L_{21kl} &  L_{22kl}
        \end{bmatrix} \\
        R_0(v\tp{\ldel_r})\tp{R_0} &= \begin{bmatrix}
            (v\tp{\ldel_r})_{00} & -(v\tp{\ldel_r})_{01} & -(v\tp{\ldel_r})_{02} \\
           -(v\tp{\ldel_r})_{10} &  (v\tp{\ldel_r})_{11} &  (v\tp{\ldel_r})_{12} \\
           -(v\tp{\ldel_r})_{20} &  (v\tp{\ldel_r})_{21} &  (v\tp{\ldel_r})_{22}
        \end{bmatrix}
    \end{align}
    Writing out the equation for $\tau^L_{00} = \tau^R_{00}(R_0)$,
    \begin{align}
        \tau^L_{00} &= \tau^R_{00}(R_0) \\
        0 &= \tau^L_{00} - \tau^R_{00}(R_0) \\
        0 &= \sum_{k=0}^2 \sum_{l=0}^2 L_{00kl} (v \tp{\ldel_r})_{kl} 
             - \sum_{k=0}^2 \sum_{l=0}^2 L_{00kl} (R_0 v \tp{\ldel_r} \tp{R_0})_{kl}
    \end{align}
    Because $\tau^L_{00}$ and $\tau^R_{00}(R_0)$ differ in only four terms,
    \begin{align}
        0 &=  2L_{0001}(v\tp{\ldel_r})_{01} + 2L_{0002}(v\tp{\ldel_r})_{02} + 2L_{0010}(v\tp{\ldel_r})_{10} + 2L_{0020}(v\tp{\ldel_r})_{20}
    \end{align}
    The velocity gradients could have arbitrary values, therefore
    \begin{equation}
        0 = L_{0001} = L_{0002} = L_{0010} = L_{0020}.
    \end{equation}
    Repeating the above process, which in this proof will be referred to as ``the procedure'',
    over every index $\tau_{ij}$ and also for the choice of $R$
    \begin{equation}
        R_1 = \begin{bmatrix}
        -1 & 0 & 0 \\
        0 &  1 & 0 \\
        0 & 0 & -1
    \end{bmatrix},
    \end{equation}
    results in
    \begin{align}
        0 &= \tau^L_{00} - \tau^R_{00}(R_0) = 2L_{0001}d_{01} + 2L_{0002}d_{02} + 2L_{0010}d_{10} + 2L_{0020}d_{20} \\
        0 &= \tau^L_{01} - \tau^R_{01}(R_0) = 2L_{0100}d_{00} + 2L_{0111}d_{11} + 2L_{0112}d_{12} + 2L_{0121}d_{21} + 2L_{0122}d_{22} \\
        0 &= \tau^L_{02} - \tau^R_{02}(R_0) = 2L_{0200}d_{00} + 2L_{0211}d_{11} + 2L_{0212}d_{12} + 2L_{0221}d_{21} + 2L_{0222}d_{22} \\
        0 &= \tau^L_{11} - \tau^R_{11}(R_0) = 2L_{1101}d_{01} + 2L_{1102}d_{02} + 2L_{1110}d_{10} + 2L_{1120}d_{20} \\
        0 &= \tau^L_{12} - \tau^R_{12}(R_0) = 2L_{1201}d_{01} + 2L_{1202}d_{02} + 2L_{1210}d_{10} + 2L_{1220}d_{20} \\
        0 &= \tau^L_{22} - \tau^R_{22}(R_0) = 2L_{2201}d_{01} + 2L_{2202}d_{02} + 2L_{2210}d_{10} + 2L_{2220}d_{20} \\
        0 &= \tau^L_{00} - \tau^R_{00}(R_1) = 2L_{0001}d_{01} + 2L_{0010}d_{10} + 2L_{0012}d_{12} + 2L_{0021}d_{21} \\
        0 &= \tau^L_{01} - \tau^R_{01}(R_1) = 2L_{0100}d_{00} + 2L_{0102}d_{02} + 2L_{0111}d_{11} + 2L_{0120}d_{20} + 2L_{0122}d_{22} \\
        0 &= \tau^L_{02} - \tau^R_{02}(R_1) = 2L_{0201}d_{01} + 2L_{0210}d_{10} + 2L_{0212}d_{12} + 2L_{0221}d_{21} \\
        0 &= \tau^L_{11} - \tau^R_{11}(R_1) = 2L_{1101}d_{01} + 2L_{1110}d_{10} + 2L_{1112}d_{12} + 2L_{1121}d_{21} \\
        0 &= \tau^L_{12} - \tau^R_{12}(R_1) = 2L_{1200}d_{00} + 2L_{1202}d_{02} + 2L_{1211}d_{11} + 2L_{1220}d_{20} + 2L_{1222}d_{22} \\
        0 &= \tau^L_{22} - \tau^R_{22}(R_1) = 2L_{2201}d_{01} + 2L_{2210}d_{10} + 2L_{2212}d_{12} + 2L_{2221}d_{21}
    \end{align}
    where for notational convenience, $d_{kl}=(v\tp{\ldel_r})_{kl}$.  From the above,
    \begin{align}
        0 &= L0001 = L0002 = L0010 = L0012 = L0020 = L0021 \\
        0 &= L1101 = L1102 = L1110 = L1112 = L1120 = L1121 \\
        0 &= L2201 = L2202 = L2210 = L2212 = L2220 = L2221 \\
        0 &= L0100 = L0102 = L0111 = L0112 = L0120 = L0121 = L0122 \\
        0 &= L0200 = L0201 = L0210 = L0211 = L0212 = L0221 = L0222 \\
        0 &= L1200 = L1201 = L1202 = L1210 = L1211 = L1220 = L1222
    \end{align}
    More elegantly,
    \begin{enumerate}
        \item $i=j \implies L_{ijkl}=0$ for all $k\neq l$.  That is, only the diagonal terms in the velocity gradient
            affect diagonal terms in the deviatic stress tensor.
        \item $i\neq j \implies L_{ijkl}=0$ if $(i,j)\neq (k,l)$ and $(i,j)\neq(l,k)$.
    \end{enumerate}
    With the above conclusions, repeating the procedure yet again for the following choices for $R$
    \begin{equation}
        R_2 = \begin{bmatrix}
            0 & -1 & 0 \\
            1 & 0 & 0 \\
            0 & 0 & 1
        \end{bmatrix}, \quad
        R_3 = \begin{bmatrix}
            1 & 0 & 0 \\
            0 & 0 & -1 \\
            0 & 1 & 0
        \end{bmatrix}, \quad
        R_4 = \begin{bmatrix}
            0 & 0 & -1 \\
            0 & 1 & 0  \\
            1 & 0 & 0
        \end{bmatrix},
    \end{equation}
    leads to
    \begin{align}
        0 &= \tau^L_{00} - \tau^R_{00}(R_2) = (L_{0000} - L_{1111})d_{00} + (L_{0011} - L_{1100})d_{11} + (L_{0022} - L_{1122})d_{22} \\
        0 &= \tau^L_{01} - \tau^R_{01}(R_2) = (L_{0110} - L_{0101})d_{10} + (L_{0101} - L_{0110})d_{01} \\
        0 &= \tau^L_{02} - \tau^R_{02}(R_2) = (L_{0220} - L_{1221})d_{20} + (L_{0202} - L_{1212})d_{02} \\
        0 &= \tau^L_{11} - \tau^R_{11}(R_2) = (L_{1100} - L_{0011})d_{00} + (L_{1111} - L_{0000})d_{11} + (L_{1122} - L_{0022})d_{22} \\
        0 &= \tau^L_{12} - \tau^R_{12}(R_2) = (L_{1221} - L_{0220})d_{21} + (L_{1212} - L_{0202})d_{12} \\
        0 &= \tau^L_{22} - \tau^R_{22}(R_2) = (L_{2200} - L_{2211})d_{00} + (L_{2211} - L_{2200})d_{11} \\
        0 &= \tau^L_{00} - \tau^R_{00}(R_3) = (L_{0011} - L_{0022})d_{11} + (L_{0022} - L_{0011})d_{22} \\
        0 &= \tau^L_{01} - \tau^R_{01}(R_3) = (L_{0110} - L_{0220})d_{10} + (L_{0101} - L_{0202})d_{01} \\
        0 &= \tau^L_{02} - \tau^R_{02}(R_3) = (L_{0220} - L_{0110})d_{20} + (L_{0202} - L_{0101})d_{02} \\
        0 &= \tau^L_{11} - \tau^R_{11}(R_3) = (L_{1100} - L_{2200})d_{00} + (L_{1111} - L_{2222})d_{11} + (L_{1122} - L_{2211})d_{22} \\
        0 &= \tau^L_{12} - \tau^R_{12}(R_3) = (L_{1221} - L_{1212})d_{21} + (L_{1212} - L_{1221})d_{12} \\
        0 &= \tau^L_{22} - \tau^R_{22}(R_3) = (L_{2200} - L_{1100})d_{00} + (L_{2211} - L_{1122})d_{11} + (L_{2222} - L_{1111})d_{22} \\
        0 &= \tau^L_{00} - \tau^R_{00}(R_4) = (L_{0000} - L_{2222})d_{00} + (L_{0011} - L_{2211})d_{11} + (L_{0022} - L_{2200})d_{22} \\
        0 &= \tau^L_{01} - \tau^R_{01}(R_4) = (L_{0110} - L_{1212})d_{10} + (L_{0101} - L_{1221})d_{01} \\
        0 &= \tau^L_{02} - \tau^R_{02}(R_4) = (L_{0220} - L_{0202})d_{20} + (L_{0202} - L_{0220})d_{02} \\
        0 &= \tau^L_{11} - \tau^R_{11}(R_4) = (L_{1100} - L_{1122})d_{00} + (L_{1122} - L_{1100})d_{22} \\
        0 &= \tau^L_{12} - \tau^R_{12}(R_4) = (L_{1221} - L_{0101})d_{21} + (L_{1212} - L_{0110})d_{12} \\
        0 &= \tau^L_{22} - \tau^R_{22}(R_4) = (L_{2200} - L_{0022})d_{00} + (L_{2211} - L_{0011})d_{11} + (L_{2222} - L_{0000})d_{22}
    \end{align}
    From the above,
    \begin{enumerate}
        \item $L_{0000} = L_{1111} = L_{2222}$.
        \item $L_{0011} = L_{0022} = L_{1100} = L_{1122} = L_{2200} = L_{2211}$.
        \item $L_{0101} = L_{0110} = L_{0202} = L_{0220} = L_{1212} = L_{1221}$.
    \end{enumerate}
    Define  $\alpha, \alpha_1,\lambda \in \R$ such that
    \begin{enumerate}
        \item $\alpha = L_{0000} = L_{1111} = L_{2222}$
        \item $\alpha_1 = L_{0101} = L_{0110} = L_{0202} = L_{0220} = L_{1212} = L_{1221}$
        \item $\lambda = L_{0011} = L_{0022} = L_{1100} = L_{1122} = L_{2200} = L_{2211}$.
    \end{enumerate}
    The procedure repeated one final time for the following choice of $R$
    \begin{equation}
        R_5 = \begin{bmatrix}
            \frac{\sqrt{2}}{2} & -\frac{\sqrt{2}}{2} & 0 \\
            \frac{\sqrt{2}}{2} &  \frac{\sqrt{2}}{2} & 0 \\
            0 & 0 & 1
        \end{bmatrix},
    \end{equation}
    and it results in
    \begin{align}
        0 &= \tau^L_{00} - \tau^R_{00}(R_5) = (\alpha/2 - \lambda/2 - \alpha_{1})d_{00} + (\lambda/2 - \alpha/2 + \alpha_{1})d_{11}.
    \end{align}
    From the above,
    \begin{align}
        0 = \frac{\alpha}{2} - \frac{\lambda}{2} - \alpha_{1} \\
        \alpha_{1} = \frac{\alpha - \lambda}{2}.
    \end{align}
    Substituting, we finally have
    \begin{enumerate}
        \item $\alpha = L_{0000} = L_{1111} = L_{2222}$.
        \item $\frac{\alpha - \lambda}{2} = L_{0101} = L_{0110} = L_{0202} = L_{0220} = L_{1212} = L_{1221}$.
        \item $\lambda = L_{0011} = L_{0022} = L_{1100} = L_{1122} = L_{2200} = L_{2211}$.
        \item Every other element $L_{ijkl}=0$.
    \end{enumerate}
    To demonstrate that this form for $L$ cannot be reduced any further, the procedure
    can be repeated for generators of $\SO(3;\R)$
    \begin{equation}
        R_6 = \begin{bmatrix}
            \cos\theta & -\sin\theta & 0 \\
            \sin\theta & \cos\theta & 0 \\
            0 & 0 & 1
        \end{bmatrix}, \quad
        R_7 = \begin{bmatrix}
            1 & 0 & 0 \\
            0 & \cos\phi & -\sin\phi \\
            0 & \sin\phi & \cos\phi
        \end{bmatrix}
    \end{equation}
    where $\theta,\phi\in \R$, which results in
    \begin{align}
        \tau^L_{ij}-\tau^R_{ij}(R_6) = 0 \\
        \tau^L_{ij}-\tau^R_{ij}(R_7) = 0.
    \end{align}
    Putting everything together, the deviatoric stress tensor must be of the form
    \begin{equation}
        \tau_{ij} = \begin{cases}
            \alpha \pd{v_i}{r_i} + \lambda \left( \tr( v \tp{\ldel_r} ) - \pd{v_i}{r_i} \right) &\text{ for $i=j$ } \\
            \frac{\alpha-\lambda}{2}\left( \pd{v_i}{r_j} + \pd{v_j}{r_i} \right) &\text{ for $i\neq j$. }
        \end{cases}
    \end{equation}
    Define
    \begin{equation}
        \mu = \frac{\alpha-\lambda}{2}
    \end{equation}
    then
    \begin{equation}
        \tau_{ij} = \begin{cases}
            2 \mu \pd{v_i}{r_i} + \lambda \tr( v \tp{\ldel_r} ) &\text{ for $i=j$ } \\
            \mu \left( \pd{v_i}{r_j} + \pd{v_j}{r_i} \right) &\text{ for $i\neq j$. }
        \end{cases}
    \end{equation}
    Because the trace of the deviatoric stress tensor is zero $\lambda = -\frac{2}{3} \mu$.
\end{proof}

\begin{thm}[Stokes Relations in Matrix-Vector Form]
    Let the \emph{shear viscosity} $\mu:\R^4\to \R$ be a flow property.
    The deviatoric stress tensor of an isotropic Newtonian fluid must be of the form
    \begin{equation}
        \tau = \mu \left( \rdel_r \tp{v} + v \tp{\ldel_r} - \frac{2}{3} (\tp{\rdel_r}v) I_3 \right).
    \end{equation}
\end{thm}
\begin{proof}
Expanding out the matrix-vector products leads to
\begin{equation}
    \tau_{ij} = \begin{cases}
        2 \mu \pd{v_i}{r_i} - \frac{2}{3} \mu \tr( v \tp{\ldel_r} ) &\text{ for $i=j$ } \\
        \mu \left( \pd{v_i}{r_j} + \pd{v_j}{r_i} \right) &\text{ for $i\neq j$. }
    \end{cases}
\end{equation}
\end{proof}

\begin{thm}[The Navier-Stokes Momentum Equation]
The momentum conservation equation of an isotropic Newtonian fluid is given
by
\begin{equation}
    -\rdel_r p + \tau \ldel_r + f_b = \rho \pd{v}{t} + \rho (v \tp{\ldel_r}) v
\end{equation}
where
\end{thm}
\begin{proof}
Beginning with the Cauchy Momentum Equation (\autoref{cauchymom}),
\begin{equation}
    \sigma \ldel_r + f_b = \rho \pd{v}{t} + \rho (v \tp{\ldel_r}) v.
\end{equation}
Substituting $\sigma = -p I_3 + \tau$,
\begin{align}
    (-p I_3 + \tau) \ldel_r + f_b &= \rho \pd{v}{t} + \rho (v \tp{\ldel_r}) v \\
    (-p I_3)\ldel_r + \tau \ldel_r + f_b &= \rho \pd{v}{t} + \rho (v \tp{\ldel_r}) v \\
    -\rdel_r p + \tau \ldel_r + f_b &= \rho \pd{v}{t} + \rho (v \tp{\ldel_r}) v.
\end{align}
\end{proof}

%\begin{thm}[Simplification of $\tau \ldel_r$]
%    a
%\end{thm}
%\begin{proof}
%\begin{equation}
%    \tau \ldel_r = \left( \mu \frac{\tau }{\mu} \right)  \ldel_r.
%\end{equation}
%Using~\autoref{lem:nablaab} and noting $\tau = \tp{\tau}$,
%\begin{align}
%    \tau \ldel_r &= \tp{ (\rdel_r \mu ) } \left( \frac{\tau}{\mu} \right)
%        + \mu \tp{\rdel_r} \left( \frac{\tau}{\mu} \right) \\
%    \tau \ldel_r &= \tp{ (\rdel_r \mu ) } \left( \frac{\tau}{\mu} \right)
%        + \mu \tp{\rdel_r} \left( \rho \pd{v}{t} + \rho (v \tp{\ldel_r}) v \right) \\
%\end{align}
%\end{proof}
