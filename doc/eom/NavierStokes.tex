\section{The Navier-Stokes Momentum Equation}
\begin{rk}
The Navier-Stokes equations, developed in the ninteenth
century by French engineer and physicist \emph{Claude-Louis Navier}
and British physicist and mathematician
\emph{Sir George Gabriel Stokes}, describe the the viscous motion
of isotropic Newtonian fluids.  The notion of a Newtonian fluid is named
after English mathematician \emph{Sir Issac Newton} who made the
observation that the deviatoric stress in many fluids can be approximated
to be proportional to the gradient of velocity.
\end{rk}

\begin{df}[Static Pressure]
    The flow property \emph{static pressure} $p:\R^4\to \R$ maps
    \begin{equation}
        (t,r)\mapsto -\frac{1}{3} \tr \sigma(t,r).
    \end{equation}
\end{df}

\begin{df}[Deviatoric Stress Tensor]
The \emph{deviatoric stress tensor} $\tau:\R^4\to M_3(\R)$
maps
    \begin{equation}
        (t,r) \mapsto \sigma(t,r) + p(t,r) I_3.
    \end{equation}
\end{df}

\begin{rk}
The deviatoric stress tensor is the deviation of the Cauchy stess tensor
from the \emph{mean hydrostatic stress tensor} $-p I_3$ also called the 
\emph{volumetric stress tensor}.
\end{rk}

\begin{df}[Newtonian Fluid]
    A fluid is \emph{Newtonian} iff the deviatoric stress tensor is symmetric and
    is a linear function of the velocity gradient.
    That is, there exist coefficients
    $L_{ijkl}\in \R$ such that
    \begin{equation}
        \tau_{ij} = \sum_{i=0}^2 \sum_{j=0}^2 \sum_{k=0}^2 \sum_{l=0}^2 L_{ijkl} (\rdel_r v)_{kl}.
    \end{equation}
    The notation $(\cdot)_{ij}$ denotes the element at the $i^\text{th}$ row and $j^\text{th}$ column of matrix $(\cdot)$.
\end{df}

\begin{rk}
A fluid is isotropic if it has the same physical properties when measured
in all directions.
\end{rk}

\begin{df}[Isotropic Newtonian Fluid]
    A Newtonian fluid is \emph{isotropic} iff for every rotation matrix $R\in \SO(3;\R)$,
    \begin{equation}
        (R\tau \tp{R})_{ij} = \sum_{i=0}^2 \sum_{j=0}^2 \sum_{k=0}^2 \sum_{l=0}^2 L_{ijkl} (R(\rdel_r v)\tp{R})_{kl}.
    \end{equation}
\end{df}

\begin{thm}[Deviatoric Stress for an Isotropic Newtonian Fluid]
Theorem to be written.
\end{thm}
\begin{proof}
    \begin{equation}
        \tau_{ij} = \sum_{i=0}^2 \sum_{j=0}^2 \sum_{k=0}^2 \sum_{l=0}^2 L_{ijkl} (\rdel_r v)_{kl}.
    \end{equation}
    Let $L_{kl}\in M_3(\R)$ denote the matrix with entries $(L_{kl})_{ij} = L_{ijkl}$.  Then
    \begin{equation}
        \tau = \sum_{k=0}^2 \sum_{l=0}^2 L_{kl} (\rdel_r v)_{kl}.
    \end{equation}
    Because $\tau$ is symmetric $L_{kl}$ is symmetric.  The fluid is isotropic, therefore
    for every $R\in \SO(3;\R)$,
    \begin{align}
        R \tau \tp{R} &= \sum_{k=0}^2 \sum_{l=0}^2 L_{kl} (R (\rdel_r) v \tp{R})_{kl} \\
        \tau &= \sum_{k=0}^2 \sum_{l=0}^2 \tp{R} L_{kl} R (R (\rdel_r) v \tp{R})_{kl} \\
        \sum_{k=0}^2 \sum_{l=0}^2 L_{kl} \underbrace{(\rdel_r v)_{kl}}_{\pd{v_l}{r_k}} &=
            \sum_{k=0}^2 \sum_{l=0}^2 \underbrace{\tp{R} L_{kl} R (R (\rdel_r) v \tp{R})_{kl}}_{B_{kl}(R)}.
    \end{align}
    Expanding out $A_{kl}$,
    %\noindent\begin{minipage}{\textwidth}
    \begin{equation*}
        \hidewidth
        A_{kl} = \begin{bmatrix}
            L_{00kl} \pd{v_0}{r_0} + L_{01kl} \pd{v_1}{r_0} + L_{02} \pd{v_2}{r_0}& L_{00kl} \pd{v_0}{r_1} + L_{01} \pd{v_1}{r_1} + L_{02kl} \pd{v_2}{r_1}& L_{00} \pd{v_0}{r_2} + L_{01kl} \pd{v_1}{r_2} + L_{02kl} \pd{v_2}{r_2} \\
            L_{01kl} \pd{v_0}{r_0} + L_{11kl} \pd{v_1}{r_0} + L_{12} \pd{v_2}{r_0}& L_{01kl} \pd{v_0}{r_1} + L_{11} \pd{v_1}{r_1} + L_{12kl} \pd{v_2}{r_1}& L_{01} \pd{v_0}{r_2} + L_{11kl} \pd{v_1}{r_2} + L_{12kl} \pd{v_2}{r_2} \\
            L_{02kl} \pd{v_0}{r_0} + L_{12kl} \pd{v_1}{r_0} + L_{22} \pd{v_2}{r_0}& L_{02kl} \pd{v_0}{r_1} + L_{12} \pd{v_1}{r_1} + L_{22kl} \pd{v_2}{r_1}& L_{02} \pd{v_0}{r_2} + L_{12kl} \pd{v_1}{r_2} + L_{22kl} \pd{v_2}{r_2}
        \end{bmatrix}
        \hidewidth
    \end{equation*}
    %\end{minipage}

    Choosing
    \begin{equation}
    R = R_0 = \begin{bmatrix}
    0 & -1 & 0 \\
    1 & 0 & 0 \\
    0 & 0 & 1
    \end{bmatrix},
    \end{equation}

\end{proof}


%    \begin{equation}
%        \tau[i,j] = \sum_{i=0}^2 \sum_{j=0}^2 \sum_{k=0}^2 \sum_{l=0}^2 L_{ijkl} \rdel_{r_l} v_k.
%    \end{equation}
%    Let $\tau[j]$ be the $j^\text{th}$ column of $\tau$.
%    Writing the summation in terms of matrix-vector products,
%    \begin{equation}
%        \tau[j] = \sum_{i=0}^2 \sum_{k=0}^2 L_{ik} \rdel_r v_k.
%    \end{equation}
%    Because the fluid is isotropic, for all $R\in \SO(3;\R)$,
%    \begin{align}
%        R\tau \tp{R} &= L \rdel_r (Rv) \\
%        R\tau \tp{R} &= L (\rdel_r v) \tp{R} \\
%        R\tau &= L (\rdel_r v)
%    \end{align}
%    and so
%    \begin{align}
%        R \tau[j] &= \sum_{i=0}^2 \sum_{k=0}^2 L_{ik} R \rdel_{r_l} v \\
%        \tau[j] &= \sum_{i=0}^2 \sum_{k=0}^2 \tp{R} L_{ik} R \rdel_{r_l} v \\
%        L_{ik} &= \tp{R} L_{ik} R.
%    \end{align}
%    Suppose $m_{ik}$ is an eigenvector of $L_{ik}$ with eigenvalue $\alpha_{ik}$.
%    Then
%    \begin{align}
%        \alpha_{ik} m_{ik} &= L_{ik} m_{ik} \\
%        \alpha_{ik} m_{ik} &= \tp{R}L_{ik}R m_{ik} \\
%        \alpha_{ik} R m_{ik} &= L_{ik} R m_{ik}.
%    \end{align}
%    $R m_{ik}$ is also an eigenvector of $m_{ik}$ with eigenvalue $\alpha_{ik}$.
%    Any unit vector on the sphere can be chosen as an eigenvector of $m_{ik}$.
%    Choose the standard basis for $\R^3$ as the set of eigenvectors.
%    Then eigendecomposition of $L_{ik}$ gives
%    \begin{equation}
%        L_{ik} = I_3 \begin{bmatrix}
%            \alpha_{ik} & 0 & 0 \\
%            0 & \alpha_{ik} & 0 \\
%            0 & 0 & \alpha_{ik}
%        \end{bmatrix} I_3 = \alpha_{ik} I_3.
%    \end{equation}
%    Therefore
%    \begin{equation}
%        \tau[j] = \sum_{i=0}^2 \sum_{k=0}^2 \alpha_{ik} \rdel_{r_l} v
%    \end{equation}
%    This means that the entire
