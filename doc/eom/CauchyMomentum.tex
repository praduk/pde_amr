\section{The Cauchy Momentum Equation}

\begin{rk}
In continuum mechanics, the traction vector is a notion
similar to that of pressure.  It is the contact force acting on an arbitrarily
small oriented area inside a body.  In the nineteenth century, a pioneering French engineer
and mathematician \emph{Augustin-Louis Cauchy} showed that all traction vectors in
any direction at a particular location in a continuum body can be represented by the
Cauchy stress tensor, a $3\times 3$ matrix valued function.
\end{rk}

\begin{rk}
The end objective of this section is to split up the momentum source term $s_{\rho v}$
into two components: body forces $f_b$ and internal surface contact forces
between control volumes $f_c$.  Examples of body forces are gravity and
electromagnetic forces that depend on the mass distribution within the body.
\end{rk}

\begin{df}[Traction]
Define a function $A$ such that for a center $c\in \R^3$, a
radius $d\in R$, and a unit normal direction $n\in \mathcal{S}_2$, $A(c,d,n)$
is a circular disc defined by the locus of points
\begin{equation}
    A(c,d,n) = \set{ r\in \R | \tp{n}(r-c)=0 \text{ and } \tp{(r-c)}{(r-c)}\le d^2 }
\end{equation}
oriented in the same direction as $n$.
    The \emph{traction} is the function $\mathfrak{t}:\R\times\R^3\times \R^3\to \R^3$ that maps
    \begin{equation}
        (t,r,n) \mapsto \lim_{d\to 0} \frac{1}{\pi d^2} F_c(t,A(r,d,n)),
    \end{equation}
    where $F_c$ is a function that maps an oriented area to the total contact force
    applied on that area.
\end{df}

\begin{thm}[Cauchy's Stress Theorem: The Stress Tensor]
For every $t\in \R$ and $r\in \R^3$ there exists a
tensor-valued function $\sigma: \R^4\to M_3(\R)$ such that
for all $n\in \mathcal{S}_2$ 
    \begin{equation}
        \mathfrak{t}(t,r,n) = \sigma(t,r) n.
    \end{equation}
\end{thm}
\begin{proof}
A non-standard beautiful proof of the existence of this stress tensor
by analyzing forces on a half-sphere can be found in~\cite{silhavy}.
This proof in addition shows that given no couple-moments (moments per unit volume)
the form of the stress tensor is symmetric and that a change of coordinates by 
rotation of elements $R\in \SO(3;\R)$ results in a rotated stress tensor $R\sigma \tp{R}$.
\end{proof}

\begin{thm}[Cauchy Momentum Equation]
    Let $f_b:\R^4\to \R$ map $(t,r)$ to be specific body forces acting on the fluid
    (e.g. $f_b(t,r)=\rho g$ for uniform gravity).
    Then
    \begin{equation}
        \label{cauchymom}
        \sigma \ldel_r + f_b = \rho \pd{v}{t} + \rho (v \tp{\ldel_r}) v.
    \end{equation}
\end{thm}

\begin{proof}
    Let $\Omega$ be any control volume and $\Gamma$ be its boudnary.  By Newton's second law,
    \begin{equation}
        \iiint_\Omega s_{\rho v} d\Omega = \text{sum of all forces acting on $\Omega$}.
    \end{equation}
    Decomposing all forces into contact forces and body forces,
    \begin{equation}
        \iiint_\Omega s_{\rho v} d\Omega = \oiint_\Gamma \mathfrak{t} d\Gamma + \iiint_\Omega f_b d\Omega.
    \end{equation}
    Using Cauchy's stress tensor,
    \begin{equation}
        \iiint_\Omega s_{\rho v} d\Omega = \oiint_\Gamma \sigma n d\Gamma + \iiint_\Omega f_b d\Omega.
    \end{equation}
    Applying the divergence theorem for rank-2 tensors,
    \begin{equation}
        \iiint_\Omega s_{\rho v} d\Omega = \iiint_\Omega \sigma \ldel_r d\Omega + \iiint_\Omega f_b d\Omega.
    \end{equation}
    By the localization theorem, 
    \begin{equation}
        s_{\rho v} = \sigma \ldel_r + f_b.
    \end{equation}
The theorem follows by substituting the above into~\autoref{convlinmom}.
\end{proof}

%\begin{rk}
%The Fubini-Tonelli Theorem gives the conditions when double integrals 
%are equivalent to iterated integrals.
%\end{rk}
%
%\begin{thm}[Fubini-Tonelli Theorem~\cite{wiki}]
%Let $X$ and $Y$ be $\sigma$-finite measure spaces.
%If $f$ is a measureable function and any one of
%the three integrals
%    \begin{align}
%        &\int_X \left( \int_Y | f(x,y)| dy \right) dx \\
%        &\int_Y \left( \int_X | f(x,y)| dx \right) dy \\
%        &\int_{X\times Y} |f(x,y)| d(x,y)
%    \end{align}
%is finite, then
%    \begin{equation}
%    \int_X \left( \int_Y f(x,y) dy \right) dx =
%    \int_Y \left( \int_X f(x,y) dx \right) dy =
%    \int_{X\times Y} f(x,y) d(x,y).
%    \end{equation}
%\end{thm}
%\begin{proof}
%See~\cite{wiki}.
%\end{proof}

%\begin{rk}
%The end objective is to split up the momentum source term $s_{\rho v}$
%into two components: body forces $f_b$ and internal surface contact forces
%between control volumes $f_c$.  Examples of body forces are gravity and
%electromagnetic forces that depend on the mass distribution within the body.
%\end{rk}

%\begin{ax}[Newton's Third Law in Continuum Mechanics]
%    Let $n\in \mathcal{S}_2$ be a unit normal direction.  Then
%    \begin{equation}
%        \mathfrak{t}(t,r,n) = -\mathfrak{t}(t,r,-n).
%    \end{equation}
%\end{ax}

%\begin{df}[Control Sphere]
%    Let $\mathfrak{B}$ be the set of all control volumes such that for
%    every element $\Omega\in \mathfrak{B}$ there exists a unique \emph{center}
%    $c\in \R^3$ and a unique \emph{radius} $d\in R$ such that for all points
%    $a\in \R^3$, $a\in \Omega$ if and only if $\tp{(a-c)}(a-c)\le d^2$.
%    The elements of $\mathfrak{B}$ are \emph{control spheres}.
%\end{df}

%\begin{df}[Control Cube]
%A control volume $\Omega$ is a \emph{control cube} iff all of the following are true
%\begin{enumerate}
%\item There exists an orthonormal basis $\{e_0, e_1, e_2\}$ for $\R^3$.
%\item There exists a center $c\in \R^3$ such that $c = c_0 e_0 + c_1 e_1 + c_2 e_2$ for some $c_0, c_1, c_2\in \R$.
%\item There exists a half-width $w\in \R$.
%\item The boundary of $\Omega$ is defined as the locus of all points $p\in \R^3$
%      such that $p= p_0 e_0+p_1 e_1+p_2 e_2$ for some $p_0,p_0,p_1\in \R$ and
%      \begin{equation}
%           \max\{ |p_0-c_0|, |p_1-c_1|, |p_2-c_2| \} = w.
%      \end{equation}
%\end{enumerate}
%\end{df}
%\begin{df}[Faces of a Control Cube]
%
%\end{df}

%\begin{df}[Traction Vector~\cite{adeeb}]
%a
%\end{df}

%\begin{thm}[Cauchy Momentum Equation]
%    Let $f:\R^4\to \R^3$ be the body forces on the fluid and the fluid property
%    $\sigma: \R^4\to M_3(\R)$ be the Cauchy stress tensor.  Then
%    \begin{equation}
%        s_{\rho v} = f + \sigma \ldel.
%    \end{equation}
%\end{thm}
%\begin{proof}
%
%\end{proof}
