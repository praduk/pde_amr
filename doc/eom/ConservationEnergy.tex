\section{Conservation of Energy}
\begin{df}[Specific Internal Energy]
    The \emph{specific internal energy} $u:\R^4\to \R$ is a flow property.
\end{df}

\begin{rk}
The internal energy of a system is the energy within a system excluding
the kinetic energy of the mean motion of the whole system.  That is
$\text{Specific Total Energy} = u + \tp{v}v/2$.
\end{rk}

\begin{df}[Internal Energy Density]
    The flow property \emph{internal energy density} $(\rho u):\R^4\to \R$ is conserved.
\end{df}

\begin{rk}
Specific internal energy is internal energy per unit mass.
Internal energy density is internal energy per unit volume.
\end{rk}

\begin{thm}[Conservation of Internal Energy]
    \begin{equation}
    \label{conenergy}
         \rho \pd{u}{t} + \rho \tp{(\rdel_r u )} v  = s_{\rho u}.
    \end{equation}
\end{thm}
\begin{proof}
    Substituting $\rho u$ into~\autoref{continuity},
    \begin{equation}
        u \pd{\rho}{t} + \rho \pd{u}{t} + \rho u \tp{\rdel_r} v  +  \tp{(\rdel_r (\rho u) )} v = s_{\rho u}.
    \end{equation}
    Simplyfing the term $\tp{(\rdel_r (\rho u) )}$ gives
    \begin{equation}
    u \pd{\rho}{t} + \rho \pd{u}{t} + \rho u \tp{\rdel_r} v + u \tp{(\rdel_r \rho )} v + \rho \tp{(\rdel_r u )} v = s_{\rho u}.
    \end{equation}
    Collecting terms
    \begin{equation}
        u \underbrace{\left( \pd{\rho}{t} + \rho \tp{\rdel_r}v + \tp{(\rdel_r \rho )} v \right)}_{
            \text{Zero by~\autoref{conmass}!}}
        + \rho \pd{u}{t} + \rho \tp{(\rdel_r u )} v  = s_{\rho u}.
    \end{equation}
\end{proof}

\begin{ax}[The First Law of Thermodynamics]
Let $\Omega$ be a control volume.  The source $s_u$ is
\begin{equation}
    \underbrace{\iiint_\Omega s_{\rho u} d\Omega}_{\text{Material Rate of Internal Energy}}
    = \underbrace{\pd{Q}{t}}_{\text{Energy Rate Added to System}}
    + \underbrace{\pd{W}{t}.}_{\text{Work Rate (Power) on System}}
\end{equation}
\end{ax}

\begin{df}[Temperature]
    The \emph{temperature} $T:\R^4\to \R$ is a flow property.
\end{df}

\begin{df}[Thermal Conductivity]
    The \emph{thermal conductivity} $k_T:\R^4\to \R$ is a flow property.
\end{df}

\begin{df}[Other Energy Sources]
    $e:\R^4\to \R$ that $(t,r)\mapsto e(t,r)$ and represents all other
    sources of heat (e.g. chemical reaction).
\end{df}

\begin{ax}[Fourier's Law of Heat Conduction]
Let $\Omega$ be a control volume with boundary $\Gamma$.
    \begin{equation}
        \pd{Q}{t} =
        -\underbrace{\oiint_\Gamma k_T \tp{\left( \rdel_r T\right)} n d\Gamma}_{\text{Thermal Conduction}} +
        \underbrace{\iiint_\Omega e d\Omega}_{\text{Heat from Reactions}}
    \end{equation}
\end{ax}

\begin{df}[Mean Velocity and Error Velocity]
    The mean velocity of a control volume $\Omega$ is a function $\bar{v}:\R\to \R^3$ that maps
    \begin{equation}
        t \mapsto \frac{\iiint_\Omega v d\Omega}{\iiint_\Omega d\Omega}.
    \end{equation}
    The error velocity is a function $\tilde{v}:\R^4 \to \R^3$ that maps
    \begin{equation}
        (t,r) \mapsto v(t,r)-\bar{v}(t).
    \end{equation}
\end{df}

\begin{thm}[Expansion of Power Term]
    \begin{equation}
        \pd{W}{t} = \iiint_\Omega (\tp{\tilde{v}} \sigma) \ldel_r + \tp{f_b} \tilde{v} d\Omega.
    \end{equation}
\end{thm}
\begin{proof}
    The work performed by a force $F:\R\to \R^3$ over a path $x:\R\to \R^3$
    is defined as
    \begin{equation}
        W = \int \tp{F(t)} dx(t).
    \end{equation}
    Differentiating both sides with respect to time gives power
    \begin{equation}
        \pd{W}{t} = F(t) \frac{dx}{dt} = \tp{F} v.
    \end{equation}
    Evalulating the power exerted by traction and body forces on the control volume,
    \begin{equation}
        \pd{W}{t} =
        \underbrace{\oiint_\Gamma \tp{\tilde{v}} \mathfrak{t} d\Gamma}_{\text{Traction Power}} +
        \underbrace{\oiint_\Omega \tp{f_b}\tilde{v} d\Omega.}_{\text{Body Force Power}}
    \end{equation}
    Replacing traction with the Cauchy stress tensor,
    \begin{equation}
        \pd{W}{t} = \oiint_\Gamma (\tp{\tilde{v}} \sigma) n d\Gamma + \iiint_\Omega \tp{f_b} \tilde{v} d\Omega.
    \end{equation}
    Using the divergence theorem,
    \begin{equation}
        \pd{W}{t} = \iiint_\Omega (\tp{\tilde{v}} \sigma) \ldel_r d\Gamma + \iiint_\Omega \tp{f_b} \tilde{v} d\Omega.
    \end{equation}
\end{proof}

\begin{thm}[Conservation of Internal Energy Expanded]
\begin{equation}
    \rho \pd{u}{t} + \rho \tp{(\rdel_r u )} v  = (k_T \tp{\left( \rdel_r T\right)} ) \ldel_r + e + \sum_{j=0}^2 \tp{\sigma_j} \rdel_{r_j} v
\end{equation}
\end{thm}
\begin{proof}
\begin{align}
    \iiint_\Omega s_{\rho u} d\Omega &= \pd{Q}{t} + \pd{W}{t} \\
    \iiint_\Omega s_{\rho u} d\Omega &= 
        \oiint_\Gamma k_T \tp{\left( \rdel_r T\right)} n d\Gamma + \iiint_\Omega e d\Omega +
        \iiint_\Omega (\tp{\tilde{v}} \sigma) \ldel_r + \tp{f_b} \tilde{v} d\Omega.
\end{align}
Using the divergence theorem,
\begin{align}
    \iiint_\Omega s_{\rho u} d\Omega &= 
        \iiint_\Omega \left( (k_T \tp{\left( \rdel_r T\right)} ) \ldel_r + e +
        (\tp{\tilde{v}} \sigma) \ldel_r + \tp{f_b} \tilde{v} \right) d\Omega.
\end{align}
In the limit as $\Omega$ scales towards zero volume, the terms with $\tilde{v}$ drop out
; in addition, because $\tilde{v}\tp{\ldel_r} = v\tp{\ldel_r}$ (the simplification of $(\tp{\tilde{v}}\sigma)\ldel_r$
is left as an exercise),
\begin{align}
    s_{\rho u} = (k_T \tp{\left( \rdel_r T\right)} ) \ldel_r + e +
        \tp{\sigma_0} \rdel_{r_0} v + \tp{\sigma_1} \rdel_{r_1} v + \tp{\sigma_2} \rdel_{r_2} v,
\end{align}
where $\sigma_0, \sigma_1, \sigma_2 \in \R^3$ are columns of $\sigma$.

\end{proof}


\begin{df}[Specific Enthalpy]
The \emph{specific enthalpy} $h:\R^4\to \R$ is a flow property that maps
    \begin{equation}
        (t,r) \mapsto u(t,r) + \frac{p(t,r)}{\rho(t,r)}.
    \end{equation}
\end{df}

\begin{thm}[Conservation of Internal Energy Expanded v2]
\begin{equation}
    \rho \pd{h}{t} + \rho \tp{(\rdel_r h )} v  = 
    \pd{p}{t} + \rho \tp{(\rdel_r p )} v +
    (k_T \tp{\left( \rdel_r T\right)} ) \ldel_r + e + \sum_{j=0}^2 \tp{\sigma_j} \rdel_{r_j} v
\end{equation}
\end{thm}
\begin{proof}
Substitute definition of enthalpy into the expanded conservation of energy equation.
\end{proof}

%\begin{ax}[First Law of Thermodynamics]
%Let source $s_u$ is
%\begin{equation}
%    s_u = \underbrace{k (\tp{\del_r} \del_r T)}_{\text{Newton's Law of Cooling}}
%    + \underbrace{\e}_{\text{Chemical Reactions}}
%    + \underbrace{\Phi}_{{\text{Specific Work on The System}}
%\end{equation}
%\end{ax}


