\section{A General Continuity Equation}

\begin{rk}
Sometimes really obvious things will be stated explicitly to
make clear the notational conventions used.
This document is designed to be read in order.
Consistent notation will be used throughout.
\end{rk}

\begin{df}[Time]
    \emph{Time} is an element $t\in \R$.
\end{df}

\begin{df}[Location]
    \emph{Location} is an element $r=(x,y,z)=(r_0,r_1,r_2)\in \R^3$.
\end{df}

\begin{df}[Control Volume]
    A \emph{control volume} is a closed, bounded,
    simply-connected subset of $\R^3$ with a piecewise smoooth boundary.
\end{df}

\begin{rk}
Sometimes notions such as ``$\R^N$'', ``a rank $n$ tensor'' and ``$M_n(\R)$''
will be stated without specifying exactly what $N$ or $n$ are when it
is clear that $N,n\in\N$.
\end{rk}

\begin{df}[Flow Property]
A \emph{flow property} is a function
$\phi:\R^4 \to M_{n\times m}(\R)$ that maps $(t,r)\mapsto \phi(t,r)$
and is continuously differentiable for all $(t,r)\in \R^4$.
\end{df}

\begin{rk}
Examples of flow properties are density $\rho:\R^4\to \R$, 
pressure $p:\R^4\to \R$, temperature $T:\R^4\to \R$, specific enthalpy $h:\R^4\to \R$,
velocity $v:\R^4\to \R^3$, specific linear momentum $\rho v:\R^4\to \R^3$, and the Cauchy stress
tensor $\sigma:\R^4\to M_3(\R)$.
\end{rk}

\begin{df}[Velocity]
\emph{The velocity} $v:\R^4\to \R^3$ is a flow property.
\end{df}

\begin{df}[Conserved Flow Property]
Let $\Omega$ be any control volume, and $\Gamma$ be the boundary of $\Omega$.
A flow property $\phi:\R^4\to \R^N$ is \emph{conserved} if and only if it satisfies the
continuity equation
    \begin{equation}
        \underbrace{\pd{}{t} \iiint_\Omega \phi(t,r) d\Omega(r)}_{\text{change within control volume}}
        =
        \underbrace{-\oiint_\Gamma \phi(t,r) \tp{v(t,r)} n(r) d\Gamma(r)}_{\text{amount entering control volume}}
        + \underbrace{\iiint_\Omega s_\phi(t,r) d\Omega(r),}_{\text{sources within control volume}}
    \end{equation}
for all control volumes $\Omega$
where $n:\R^3\to \R^3$ is the surface normal of $\Gamma$ oriented positive towards
the exterior of $\Omega$ and
\emph{the source} of the conserved flow property $\phi$ is the function
$s_\phi:\R^4 \to \R^N$ that maps $(t,r)\mapsto s_\phi(t,r)$.
\end{df}

\begin{rk}[Vector Differential Operators]
In an effort to be able to use the differential operator $\nabla$ in any place you
 could a vector, for example in a dot product (divergence), a cross product (curl)
or even a matrix-vector product, the symbols $\rdel$ and $\ldel$ are introduced
so that equations can be written with aesthetic precision.
\end{rk}

\begin{df}[The operator $\rdel$]
    Let $g:\R^n\to \R^m$ map $a \to g(a)$.
    The operator $\rdel_a$ is a vector of partial differential operators,
    \begin{equation}
        \rdel_a = \begin{bmatrix}
            \pd{}{a_0} \\
            \pd{}{a_1} \\
            \vdots \\
            \pd{}{a_n}
        \end{bmatrix}
    \end{equation}
    so that
    \begin{equation}
        \rdel_a \tp{g} = \begin{bmatrix}
            \pd{}{a_0} \\
            \pd{}{a_1} \\
            \vdots \\
            \pd{}{a_n}
        \end{bmatrix} \begin{bmatrix}
            g_0 & g_1 & \cdots & g_m
        \end{bmatrix}
        = \begin{bmatrix}
            \pd{g_0}{a_0} & \pd{g_1}{a_0} & \cdots & \pd{g_m}{a_0} \\
            \pd{g_0}{a_1} & \pd{g_1}{a_1} & \cdots & \pd{g_m}{a_1} \\
            \vdots & \vdots & \ddots & \vdots \\
            \pd{g_0}{a_n} & \pd{g_1}{a_n} & \cdots & \pd{g_m}{a_h}
        \end{bmatrix}
        =\tp{\left( \pd{g}{a} \right)}.
    \end{equation}
\end{df}

\begin{rk}
The gradient of a scalared valued function $g$ is normally
written $\nabla g$. $\rdel_r$ extends this concept to vector valued
functions and it specifies exactly wich variables ($r$) the partials
are taken with respect to.
\end{rk}


\begin{rk}
The operator $\rdel$ acts on a function that is syntactically adjacent to the right side of $\rdel$. 
For notational convenicence the operator $\ldel$ is introduced so that it acts exactly like $\rdel$ but
on functions that are syntactically adjacent to the left side $\ldel$.  For example, for $\alpha\in \R$
and $f:\R^n\to \R^m$ mapping $a\mapsto f(a)$,
\begin{align}
    \alpha \tp{\left(\pd{f}{a}\right)} &= \alpha \rdel_a \tp{f} = \tp{(f \tp{\ldel_a})} \alpha = \tp{\left(\pd{f}{a}\right)} \alpha \\
    \alpha \pd{f}{a} &= \alpha \tp{(\rdel_a \tp{f})} = f \tp{\ldel_a} \alpha = \pd{f}{a} \alpha \\
    \rdel_a \cdot f &= \tp{\rdel_a} f = \tp{f} \ldel_a = f \cdot \ldel_a.
\end{align}
This newly introduced notation makes it much nicer to write tortuous expressions that involve matrix
valued functions such as $\tp{(\tp{\rdel_a} \tp{A})}$ as $A \ldel_a$ instead.
Excluding associativity on the side it is acting on and commutativity,
the new operators can be used much like ordinary vectors.
\end{rk}

\begin{thm}[Divergence Theorem for Rank-2 Tensors over $\R$]
    Let $\Omega$ be a subset of $\R^3$ which is compact and has
    a piecewise smooth boundary $\Gamma$.
    Let $A:\R^3 \to M_{N\times 3}(\R)$ be continuously differentiable
    on a neighborhood of $\Omega$.  Then
    \begin{equation}
        \oiint_\Gamma A(r) n(r) d\Gamma(r)
        = \iiint_\Omega A(r) \ldel_r d\Omega(r),
    \end{equation}
    where $n:\R^3\to \R^3$ is the outward pointing normal of $\Gamma$.
\end{thm}
\begin{proof}
    If $N=1$, the theorem statement is identical to the divergence theorem for vectors over $\R$.
    That is for $a:\R^3 \to \R^3$, the theorem reads
    \begin{equation}
        \oiint_\Gamma \tp{a(r)} n(r) d\Gamma(r)
        = \iiint_\Omega  \tp{a(r)} \ldel_r d\Omega(r).
    \end{equation}
    Writing out the divergence theorem for rank-2 tensors row by row results in $N$
    statements of the divergence theorem for vectors over $\R$.
    Similarily,  it is easy to generalize the divergence theorem to rank-n tensors over $\R$,
    e.g. see~\cite{riley}.
\end{proof}

\begin{rk}
    For brevity, the evaluation of functions such as $\phi(t,r)$ will be simply written
    as $\phi$ when the meaning is clear.
\end{rk}

\begin{thm}[The Continuity Equation as a PDE]
Iff $\phi$ is a conserved flow property,
\begin{equation}
    \label{continuity0}
    \pd{\phi}{t} + ( \phi \tp{v} ) \ldel_r = s_\phi.
\end{equation}
\end{thm}

\begin{proof}
    Starting with the continuity equation,
    \begin{equation}
        \pd{}{t} \iiint_\Omega \phi d\Omega(r) =
        -\oiint_\Gamma \phi \tp{v} n d\Gamma(r)
        + \iiint_\Omega s_\phi d\Omega(r),
    \end{equation}
    and applying the divergence theorem for rank-2 tensors,
    \begin{equation}
        \pd{}{t} \iiint_\Omega \phi d\Omega(r) =
        -\iiint_\Omega \left(\phi \tp{v} \right) \ldel_r d\Omega(r)
        + \iiint_\Omega s_\phi d\Omega(r).
    \end{equation}
    As a special case of the Reynolds Transport Theorem where $\Omega$ isn't
    a function of time (or can be proved using the definition of the integral),
    the time derivative on the left-hand side can be moved into the integral and
    \begin{equation}
        0 = \iiint_\Omega \left( \pd{\phi}{t} +
        \left(\phi \tp{v} \right) \ldel_r
        - s_\phi \right) d\Omega(r).
    \end{equation}
    Because the above equation is true for all $\Omega$, the theorem
    statement follows by applying the localization theorem~\cite{wiki}.
\end{proof}

\begin{lem}
Let $a:\R^3\to \R^N$ and $b:\R^3\to \R^3$.  Then
\begin{equation}
    \label{lem:nablaab}
    (a\tp{b})\ldel_r = a \tp{\rdel_r} b  + a \tp{\ldel_r} b.
\end{equation}
\end{lem}
\begin{proof}
    \begin{align}
        (a\tp{b})\ldel_r &= \begin{bmatrix}
            b_x a & b_y a & b_z a
        \end{bmatrix} \ldel_r \\
        &= \pd{}{x} (b_x a) + \pd{}{y} (b_y a) + \pd{}{z} (b_z a) \\
        &= \pd{b_x}{x} a + b_x \pd{a}{x} +
           \pd{b_y}{y} a + b_y \pd{a}{y} +
           \pd{b_z}{z} a + b_z \pd{a}{z} \\
        &= \left( \pd{b_x}{x} + \pd{b_y}{y} + \pd{b_z}{z} \right)a +
           \left(b_x \pd{a}{x} + b_y \pd{a}{y} + b_z \pd{a}{z} \right) \\
        &= a ( \tp{\rdel_r} b ) + \begin{bmatrix}
            \pd{a}{x} & \pd{a}{y} & \pd{a}{z}
        \end{bmatrix} b \\
        &= a (\tp{\rdel_r} b) + a \tp{\ldel_r} b.
    \end{align}
\end{proof}

\begin{thm}[The Continuity Equation as a PDE, Final Form]
Iff $\phi$ is a conserved flow property,
\begin{equation}
    \label{continuity}
    \pd{\phi}{t} + \phi \tp{\rdel_r} v +  \phi \tp{\ldel_r} v = s_\phi.
\end{equation}
\end{thm}
\begin{proof}
    Direct consequence of \autoref{lem:nablaab} and \autoref{continuity0}.
\end{proof}

